%!TEX program = xelatex
\documentclass[11pt,a4paper,sans,UTF8]{moderncv}


\usepackage{amsmath}

% moderncv themes
\moderncvstyle{banking}
\definecolor{color0}{rgb}{0,0,0}% black
\definecolor{color1}{rgb}{0.95,0.20,0.20}% red
\definecolor{color2}{rgb}{0.0,0.0,0.0}% dark grey



\AtBeginDocument{
\hypersetup{colorlinks,urlcolor=blue}
}


%%%%%%%%%%%%%%%%%%%%%%%%%%%%%%%%%%
\renewcommand*{\namefont}{\fontsize{50}{52}\mdseries\upshape}
%%%%%%%%%%%%%%%%%%%%%%%%%%%%%%%%%%

\usepackage[top=2cm, bottom=2cm, left=2.8cm, right=2.8cm]{geometry}
%\usepackage[scale=0.75]{geometry}


\usepackage{xeCJK}

\setCJKmainfont{宋体}
\setCJKsansfont{黑体}
\setCJKmonofont{楷体}

\usepackage{xunicode}

% personal data
\firstname{王}
\familyname{跃}
%\title{Resumé title (optional)} 
\address{四川省成都金牛区二环路北一段 111号}{西南交通大学九里堤校区}{邮编:610031}
\mobile{+86~152~8106~5825}                            
\email{shrimpek\_kk@163.com}                  
\homepage{unkeltao.github.com}
%\extrainfo{additional information}   
      
%\quote{Some quote (optional)}      

\begin{document}
\makecvtitle
%\maketitle

\section{个人信息}
\cvitem{姓名}{王跃}
\cvitem{年龄}{23(1990-11-2)}
\cvitem{毕业时间}{2016-6}
%\cvitem{就业意向}{研发工程师}
\photo[64pt][0.4pt]{jianli}  
\section{教育经历}
\cventry{2009--2013}{计算机科学与技术专业}{\textsf{西南交通大学(211、特色985工程)}}{四川省成都市}
{工学学士}
{保送至西南交通大学攻读研究生}
{}


~\\
\cventry{2013--至今}{计算机技术专业}{\textsf{西南交通大学}}{四川省成都市}
{工学硕士}
{}

%\section{Master thesis}
%\cvitem{title}{\emph{Title}}
%\cvitem{supervisors}{Supervisors}
%\cvitem{description}{Short thesis abstract}
\section{相关经验}
\subsection{项目经验}
\cventry{2013-10}{Hadoop,数据挖掘分析}{\textsf{北京市出租车GPS预测}}{}{}{
	这个项目实现利用北京市一个月的所以出租车GPS位置数据(50G)与北京市路网信息图进行简要建模,向用户告知所在位置5分钟内打到车的概率与第一辆出租车经过的预计时间。
	\begin{description}
	%	\item[\textsf{项目主页}] https://gitcafe.com/UESTC\_ACM/cdoj
		\item[\textsf{开发平台}] Jdk1.7, Hadoop 1.1.0, c/c++
		\item[\textsf{开发工具}] Eclipse, Sqllite, Sublime Text
		\item[\textsf{相关框架}] MapReduce
		\item[\textsf{个人职责}] 负责前期出租车GPS位置数据的预处理和提取出相关有用的信息(MapReduce)与利用已经导入数据库中的信息进行快速查询给出结果(C/C++)。
	\end{description}
}

~\\
\cventry{2013-5}{本科毕业设计}{\textsf{基于MapReduce的模糊K-Means设计与实现}}{}{}{
	\begin{description}
		\item{\textsf{开发平台}} JDK1.7, Hadoop 1.0.4
		\item{\textsf{开发工具}} Sublime Text, Eclipse, Matlab
		\item 利用Mapreduce框架和Hadoop平台,设计出能够处理大数据的模糊K-Means聚类方法。
	\end{description}
}
~\\
\cventry{2012-2013}{Web开发,JavaEE}{\textsf{西南交大上机考试系统}}{}{}{
	这个项目实现一个在线考试系统,能够进行在线判题并给出结果,用来提供在线测评服务。此系统用于学校 大学生C/C++、Java上机考试以及西南交大研究生复试上机测试。
	\begin{description}
	%	\item[\textsf{项目主页}] https://gitcafe.com/UESTC\_ACM/cdoj
		\item[\textsf{开发平台}] Jdk1.7, J2ee platform
		\item[\textsf{开发工具}] MyEclipse, Sqlserver, Chrome
		\item[\textsf{相关框架}] Struts2, Spring, Hibernate, jQuery
		\item[\textsf{个人职责}] 参与数据库设计,以及后台的相关工作。
	\end{description}
}
\subsection{活动经历}
\cventry{2010--2011}{组员,Web开发}{\textsf{学校星网开发小组}}{}{}{
	\begin{itemize}
		\item{\textsf{主要工作}} 开发与维护学校星网论坛。
	\end{itemize}
}
~\\
\cventry{2011--2013}{成员,Web开发}{\textsf{西南交大网站中心}}{}{}{
	\begin{itemize}
		\item{\textsf{主要工作}} 重构或开发网络中心相关业务管理平台。
	\end{itemize}
}
~\\
\cventry{2011--2013}{队员,算法学习}{\textsf{西南交大ACM校队}}{}{}{
	\begin{itemize}
		\item 学习各种算法如数论,动态规划,常用数据结构,拥有较强的代码能力。
		\item 在 ACM/ICPC 的竞赛中,积累了大量团队合作解题的经验,有很强的团队意识。
	\end{itemize}
}
\section{荣誉}
\subsection{ACM-ICPC}
\cvline{2013-6}{第二届西南地区暨第五届四川省ACM/ICPC程序设计大赛 \textsf{金奖}}
\cvline{2012-6}{第一届西南地区暨第四届四川省ACM/ICPC程序设计大赛 \textsf{一等奖}}
\cvline{2012-5}{第 37 届 ACM-ICPC 亚洲区预选赛(浙江金华赛区)邀请赛 \textsf{铜牌}}
\subsection{其他}
\cvline{2012}{校级综合奖学金\textsf{一等}}
\section{专业技能}
\subsection{语言技能}
\cvitem{\textsf{全国大学英语四级考试}}{通过}

\subsection{计算机技能}
\cvitem{\textsf{操作系统}}{linux(Ubuntu), Windows}
\cvitem{\textsf{版本控制}}{Subversion, Git}
\cvitem{\textsf{编程语言}}{C, C++, Java(Java EE), PHP}
\cvitem{\textsf{开发工具}}{Vim, Eclipse/MyEclipse, Sublime Text}
\cvitem{\textsf{其它工具}}{MySQL, \LaTeX}
%\cvitem{\textsf{相关证书}}{全国计算机等级考试四级网络工程师}

\subsection{其它技能}
\cvitem{\textsf{问题解决}}{能在较短时间内独立思考、周密分析问题并形成解决问题的思路;由于经常参加程序竞赛,提升了算法设计能力和编程技巧、养成了良好的协作精神、锻炼了心理素质和临场应变能力}

\end{document}